\section{Rollback Protection}
As mentioned before, CCF only provides rollback protection for its nodes. They cannot be rolled back, because a node can never restart in CCF, but has to enter as a completely new node. That means a node can be killed, but never cause a rollback attack on the system. In contrast, the ledger can be rolled back, because it is stored in the persistent storage outside the Trusted Execution Environment and CCF does not provide an additional mechanism to detect rollback attacks. This makes CCF vulnerable and a risk for applications with high requirements of security.\\
 Nimble also stores its state in an existing storage service outside the TEE to use its API, which also makes the TCB smaller. However, in contrast to CCF, it protects this state from being compromised. Nimble, on the other hand, has extended protection against these attacks. To protect rollback attacks they first must be detected. Therefore Nimble presents three categories of rollback attacks:
	\subsubsection*{Stale responses:} Stale responses happen when old data is given by the host although there is already newer data. This can easily happen due to the external storage service that cannot be trusted. To prevent stale responses, the state is stored twice. At first, it is encrypted and stored in an external storage service, then the state is also stored in the append-only ledger. When an application wants to read the state of the system, it reads both, the one from the storage service, and the one from the ledger, and can detect an attack, when they are not the same. To guarantee liveness a counter is also stored as well in the storage service as on the ledger. The counter on the storage service is always incremented by one and then copied by the ledger. If the system fails before the new state can be appended to the ledger, this can be recognized, as the counter of the ledger is one lower than the one of the storage service. Accordingly, the new state can be added to the ledger, when the system restarts. Nimble uses a linearizable append-only ledger. Linearizability is a criterion that provides strong consistency and is realized via the endorsers.\\
	 Remember that an endorser only stores the tail of the ledger, so if data is read from the ledger it can be compared with the endorser. In case it is not the same, a rollback attack did happen or the system failed before the endorser could store the tail of the hash chain. Accordingly, the ledger cannot be rolled back, because the state is provided by the endorser, which does not have a previous state to roll back to. 
	\subsubsection*{Synthesized requests:} Synthesized requests mean that the provider sends requests that are not sent by the application and stores them. This is handled via signing. The application signs the state and then appends it to the ledger. When the application reads a state from the ledger it then can recognize whether it is a real state with a signature or a synthesized request from a malicious provider.
	\subsubsection*{Replay:} When a provider applies older requests to the storage this is called replay. Therefore the position of the stored state in the ledger is also stored with a signature. Accordingly, the application can check if this signed position matches with the position it is currently stored in the ledger. So Nimble detects the rollback attack and prevents it. According to this, also a replay does not maliciously affect the system.\\
	
%\subsection{TCB}