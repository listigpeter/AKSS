\section{Background}
Before I start with an example of a confidential cloud service some definitions must be clear to understand the following parts.
\subsection{Trusted Execution Environments}
A Trusted Execution Environment (TEE) is a hardware based component where critical code can be runned inside a trusted part that is called enclave. The code inside this Enclave is hard to modify by malicious parties and can thus be trusted more and is therefore often be used for confidential computing.  But of course it is not impossible to change code inside a TEE, so this eventualities should also be kept in mind. Through the concept of remote attestation it is possible to make sure that the correct code runs inside the TEE~\cite{remoteAttestation}. \\ %TODO: find a paper for TEEs
Trusted Execution Environments can be realized either on a hardware processor like Intel SGX~\cite{sgx} or in a virtual machine like AMD SEV-SNP~\cite{amd}.%TODO: find paper
\subsection{Ledger}
A ledger is a digital register that is made to document transactions or other structural data. It is often used for blockchains or confidential computing.
\subsection{Rollback Attacks}
There exist many different attacks against cloud services like forking attacks~\cite{forkingAttacks}, side channel attacks~\cite{sideChannel} or rollback attacks~\cite{Rollback}. These are all important attacks that should be detected and protected by a confidential cloud service, but in this paper I will focus on rollback attacks.\\
In this attack malicious parties safe an older version of the system, restart it, and apply this older version as a new state. So they "roll back" the state of the system. This can be useful for guessing encryption keys. For these keys there often exist some limitations of guesses, so the key cannot get brute forced. The problem is with rolling back the state of the system a malicious member can go back to the state before the first try and guess again. So brute forcing the encryption key is still possible by using these attacks. 