\section{Background}
In this section, I will give the necessary definitions and background to understand the following parts. 
\subsection{Trusted Execution Environments}
A Trusted Execution Environment (TEE) is a hardware-based component where critical code can be run inside a trusted part that is called an enclave. The code and data are confidentiality and integrity protected by malicious parties, hypervisor and the cloud provider.  But of course, it is not impossible to change code inside a TEE, so these eventualities should also be kept in mind. Through the concept of remote attestation, it is possible to make sure that the correct code runs inside the TEE~\cite{remoteAttestation}.  The basic concept is that an attestation key exists with which the TEE can sign its binary and other things, called the quote. A client can then verify this quote. \\ %TODO: find a paper for TEEs
Trusted Execution Environments can be realized either on a hardware processor like Intel SGX~\cite{sgx} or in a virtual machine like AMD SEV-SNP~\cite{amd}.%TODO: find paper
\subsection{Distributed Ledger Technology (DLT)}
A ledger is a digital register that is made to document transactions or other structural data. The Distributed Ledger Technology (DLT) contains multiple redundant digital ledgers that are decentralized and collect data about transactions in a network and sign them with a cryptographic signature~\cite{ledger}. It is often used for blockchains or confidential computing. Most ledgers in the cloud services are append-only ledgers to guarantee the properties of auditability and trustworthiness. This means that data can be written only once in the ledger, but read multiple times. In addition, entries in the ledger cannot be deleted. One use of such a distributed ledger technology (DTB) is a blockchain that is mostly used in the bitcoin technology.
\subsection{Rollback Attacks}
There exist many different attacks against cloud services like forking attacks~\cite{forkingAttacks}, side channel attacks~\cite{sideChannel} or rollback attacks~\cite{Rollback}. These are all important attacks that should be detected and protected by a confidential cloud service, but in this paper I, will focus on rollback attacks.\\
In this attack, malicious parties save an older version of the system, restart it, and apply this older version as a new state. So they "roll back" the state of the system. Such an attack can lead to severe consequences. If a bank stores its transactions in a cloud storage, a compromised user can roll back whole transactions. Another example is breaking car keys. %TODO: write more about it