\section{Background}
In this section, I will give the necessary definitions and background to understand the following parts. 
\subsection{Trusted Execution Environments}
A TEE is a processor, e.g. Intel SGX~\cite{sgx}, or VM-based, e.g. AMD SEV-SNP~\cite{amd}, component where critical code can be run inside a trusted part that is called an enclave. The code and data are confidentiality and integrity protected from malicious parties, hypervisor, and the cloud provider.  But of course, it is not impossible to change code inside a TEE. Through the concept of remote attestation, it is possible to make sure that the correct code runs inside the TEE~\cite{remoteAttestation}.  The basic concept is that an attestation key exists with which the TEE can sign its binary and remote entities with a so-called quote, which includes the hash of the initial code and data. A client can then verify this quote. \\ %TODO: find a paper for TEEs
As Trusted Execution Environments provide confidentiality and integrity, they are mostly used as a base component of a CCS. Nevertheless, it is important to note that a TEE can also get compromised and therefore violates the CIA properties.
\subsection{Distributed Ledger Technology}
A ledger is a digital register that records transactions in blocks. Distributed Ledger Technology (DLT) contains multiple redundant digital ledgers that are decentralized and collect data about transactions in a network and sign them with a cryptographic signature~\cite{ledger}. A distributed ledger technology is used e.g. as the base technology for the Cryptocurrency Bitcoin.\\
 However, DLTs are also used for CCS, because they provide confidentiality. Therefore most ledgers are append-only to guarantee the properties of auditability and trustworthiness. This means that data can be written only once in the ledger, but read multiple times. In addition, entries in the ledger cannot be deleted. 
\subsection{Rollback Attacks}
Cloud services are vulnerable to several attacks like forking attacks~\cite{forkingAttacks}, side-channel attacks~\cite{sideChannel}, or rollback attacks~\cite{Rollback}. In essence, a CCS should protect against all of these attacks but in the scope of this paper, I only focus on rollback attacks.\\
In this attack, malicious parties save an older version of the system, restart it, and apply this older version as a new state, rolling back the state of the system. Such an attack can lead to severe consequences. If a bank stores its transactions in a cloud storage, a compromised user can roll back whole transactions. Another example is breaking car keys~\cite{rolljam}. Key fobs send unique signals to the car that opens when it gets the right signal. A hacker can block the signal, record it, and send it to the car again. Because the first signal did not reach the car, it can be opened with that signal.\\%TODO: write more about it
Of course, rollback attacks can also happen on a CCS and therefore it is mandatory to protect against this kind of attack. Otherwise, whole transactions or configurations can be undone which violates the integrity of the CCS.