\section{Discussion}
\label{challenges:ccf}
As mentioned before CCF does not have a rollback detection, because when an attack happens to nodes it cannot affect the system. The problems are that they (1) make the assumption that the code inside the TEE cannot be changed and (2) that they put their ledger into the persistent storage outside the TEEs and  it is therefore not protected for a rollback attack.\\
Solutions for these challenges exist in Nimble.\\
Another challenge is the TCB. If both systems are compared, CCF contains of about 55.5 thousand lines of code in the TEE, while Nimble has only 2.3 thousand. With a bigger TCB, the code inside the TEE gets more vulnerable. On the other hand, CCF is much more powerful. While Nimble is specialized on rollback attacks, CCF also provides the possibility to use it for multiparty applications.  Despite it would be more efficient if confidential cloud services do not try to put that much code in the TCB to not depend on other services, but reduce the TCB for their field of application. But this leads to the problem of performance. Nimbles throughput is lower than the C++ implementation of CCF. So the right relation of as small TCB, but also high throuput has to be found and decided depending on the application.\\
Both of the services assume that TEEs are not manipulated and the code that runs inside them is alway correct. The problem is that according to the current state of research this cannot be guaranteed. A Trusted Execution Environment can suffer from different attacks like side channel attacks or physical attacks. Due to this it can happen that the data inside the TEE is changed and the code inside the TEE does not fulfill the condition of integrity or confidentiality. Accordingly, both CCF and Nimble would violate the CIA criteria. 