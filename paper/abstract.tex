\section*{Abstract}	
 Nowadays, many services are being outsourced to the cloud, because of achieving more flexibility and scalability. The problem is that cloud services are very unsecure snd therefore not usable for applications with high security requirements. One solution is using Trusted Execution Environments (TEEs) to provide more safety. Therefore many services exist that integrated TEEs and other mechanisms to make the cloud environment confidential. These systems are called Confidential Cloud Services (CCS).\\
  In this paper I compare two of these services, the Confidential Consortium Framework (CCF) and Nimble. They both fulfill the requirements of the CIA triad which are confidentiality, integrity and high availability. Especially when it comes to integrity, they both differ from each other. While CCF does not have protection against rollback attacks and is therefore vulnerable for them, Nimble specializes on detecting and protecting that kind of attack. Nimble also has the feature of keeping the Trusted Computing Base (TCB) as small as possible. This paper shows that both systems have strengths and weaknesses. %In comparison with CCF, the TCB of Nimble is much smaller and therefore less vulnerable. CCF, on the other hand, has better performance than Nimble, which is also an important aspect of cloud services. Both systems have the weakness of assuming that first, TEEs cannot get compromised, and second, ignoring other attacks that can happen like forking attacks, side-channel attacks or physical attacks. 
  Accordingly, both systems have good approaches in different subareas that are all very important for cloud computing, but nevertheless harbor high risks for applications with high requirements of security.\\